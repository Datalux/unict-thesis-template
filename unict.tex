\documentclass[12pt,a4paper]{report}
\usepackage[utf8]{inputenc}
\usepackage[italian]{babel}
\usepackage{geometry}
\usepackage{amsmath}
\usepackage{amsfonts}
\usepackage{amssymb}
\usepackage{graphicx, array}
\usepackage{fancyhdr}
\pagestyle{fancy}

\usepackage{float}
\usepackage{subcaption}
\usepackage{amsmath}
\usepackage{amsfonts}
\usepackage{amssymb}
\usepackage{graphicx}

\newcommand{\fskip}{
\bigskip
\bigskip
\bigskip
}

\newenvironment{dedi}
  {%\clearpage           % we want a new page          %% I commented this
   \newpage
   \thispagestyle{empty}% no header and footer
   \vspace*{\stretch{1}}% some space at the top
   \itshape             % the text is in italics
   \raggedleft          % flush to the right margin
  }
  {\par % end the paragraph
   \vspace{\stretch{3}} % space at bottom is three times that at the top
   \clearpage           % finish off the page
  }
  
  
\newcommand{\utitlee}[3]{
\begin{center}
\includegraphics[scale=0.5]{images/logo.png}\\
\bigskip
{\LARGE UNIVERSITÀ DEGLI STUDI DI CATANIA}\\
{\small DIPARTIMENTO DI MATEMATICA E INFORMATICA}\\ 
{\footnotesize CORSO DI LAUREA IN INFORMATICA}\\
\noindent\rule{12cm}{0.4pt}
\end{center}}

\newcommand{\universita}[3]{
\begin{center}
\includegraphics[scale=0.5]{images/logo.png}\\
\bigskip
{\LARGE #1}\\
{\small #2}\\ 
{\footnotesize #3}\\
\noindent\rule{12cm}{0.4pt}
\end{center}}


\newcommand{\studente}[1]{
\fskip
\begin{center}
\large #1\\
\fskip
\end{center}

}


\newcommand{\ttitle}[1]{
\begin{center}
\Large
\textbf{#1}

\fskip\fskip
\end{center}
}


\newcommand{\tipo}[1]{
\begin{center}

\noindent\rule{4cm}{0.4pt}\\
\large #1\\
\vspace*{-0.4\baselineskip}
\noindent\rule{4cm}{0.4pt}

\end{center}
}

\newcommand{\relatore}[1]{
\hspace*{\fill}\textbf{Relatore:}\\
\hspace*{\fill}\large #1
}

\newcommand{\correlatore}[1]{
\hspace*{\fill}\textbf{Correlatore:}\\
\hspace*{\fill}\large #1
}

\newcommand{\data}[1]{
\begin{center}
\noindent\rule{12cm}{0.4pt}\\
\large #1
\end{center}
}

\newcommand{\dedica}[1]{
\begin{dedi}
    #1
\end{dedi}
}



\pagenumbering{arabic}

\begin{document}


% QUI È POSSIBILE SCEGLIERE I MARGINI
\newgeometry{top=3cm, bottom=4cm}

\thispagestyle{empty}

% QUI È POSSIBILE INSERIRE UNIVERSITÀ, DIPARTIMENTO E CORSO DI LAUREA
\universita
{UNIVERSITÀ DEGLI STUDI DI CATANIA}
{DIPARTIMENTO DI MATEMATICA E INFORMATICA}
{CORSO DI LAUREA IN INFORMATICA}

% QUI È POSSIBILE INSERIRE IL NOME DELLO STUDENTE

\studente{Giuseppe Criscione}

% QUI È POSSIBILE INSIERIRE IL TITOLO DELLA TESI

\ttitle{Il titolo della tesi}

\tipo{PROVA FINALE}

\fskip\fskip

% QUI È POSSIBILE INSIERIRE IL NOME DEL RELATORE

\relatore{Prof. Re Latore}

% Nel caso sia presente anche un correlatore è possibile aggiungere
% \correlatore{NOME CORRELATORE}

\fskip

% QUI È POSSIBILE INSERIRE LA DATA DI LAUREA
\data{Novembre (si spera) 2019}

\pagenumbering{roman}

% QUI È POSSIBILE INSERIRE UNA DEDICA

\dedica{La mia dedica.}

% INDICE DEI CONTENUTI

\tableofcontents

% INDICE DELLE TABELLE

\listoftables

% INDICE DELLE IMMAGINI

\listoffigures

\cleardoublepage

\pagenumbering{arabic}

% QUI È POSSIBILE INSERIRE INTRODUZIONE, CAPITOLI E CONLCUSIONI

\chapter{Introduzione}


Lorem ipsum dolor sit amet, consectetur adipiscing elit. Donec suscipit posuere felis nec accumsan. Donec ultricies laoreet tortor at sollicitudin. Aenean ac congue ante. Sed quam dolor, vulputate ac dui porttitor, ullamcorper facilisis lectus. Cras quis pretium elit. Proin malesuada risus quis hendrerit eleifend. Etiam sit amet lacinia purus, nec semper eros. Aliquam ultrices erat id viverra convallis. Phasellus in finibus quam, id placerat nisl. Nullam odio felis, tristique ac diam sed, blandit rutrum neque. Integer accumsan orci orci, ac auctor lacus tempor vel. Morbi suscipit massa in vulputate tincidunt. Fusce laoreet ornare elit, vitae tempor mi dapibus sed. Duis aliquam varius porta. Donec tempus viverra ipsum, maximus volutpat mauris pulvinar ut.

Duis sit amet libero aliquam leo lacinia dignissim non non lectus. Donec cursus nulla eu justo aliquet, et rhoncus est mollis. Nunc at ullamcorper nulla. Suspendisse mattis molestie tempus. Nunc eget semper justo, vitae placerat turpis. Aenean sodales orci arcu, at commodo ligula viverra non. Quisque finibus euismod lacus auctor tempus. Integer id urna non nunc tempus iaculis in sit amet tellus. Duis vel facilisis nisi. Nullam interdum leo ut pulvinar volutpat. Aenean lobortis lectus nec consequat volutpat. Mauris gravida et ex sit amet pharetra. Cras euismod euismod massa, at condimentum augue cursus quis. Nullam sed ante sit amet libero hendrerit posuere eget a dolor. 

\chapter{Titolo del capitolo 1}


Lorem ipsum dolor sit amet, consectetur adipiscing elit. Donec suscipit posuere felis nec accumsan. Donec ultricies laoreet tortor at sollicitudin. Aenean ac congue ante. Sed quam dolor, vulputate ac dui porttitor, ullamcorper facilisis lectus. Cras quis pretium elit. Proin malesuada risus quis hendrerit eleifend. Etiam sit amet lacinia purus, nec semper eros. Aliquam ultrices erat id viverra convallis. Phasellus in finibus quam, id placerat nisl. Nullam odio felis, tristique ac diam sed, blandit rutrum neque. Integer accumsan orci orci, ac auctor lacus tempor vel. Morbi suscipit massa in vulputate tincidunt. Fusce laoreet ornare elit, vitae tempor mi dapibus sed. Duis aliquam varius porta. Donec tempus viverra ipsum, maximus volutpat mauris pulvinar ut.

Duis sit amet libero aliquam leo lacinia dignissim non non lectus. Donec cursus nulla eu justo aliquet, et rhoncus est mollis. Nunc at ullamcorper nulla. Suspendisse mattis molestie tempus. Nunc eget semper justo, vitae placerat turpis. Aenean sodales orci arcu, at commodo ligula viverra non. Quisque finibus euismod lacus auctor tempus. Integer id urna non nunc tempus iaculis in sit amet tellus. Duis vel facilisis nisi. Nullam interdum leo ut pulvinar volutpat. Aenean lobortis lectus nec consequat volutpat. Mauris gravida et ex sit amet pharetra. Cras euismod euismod massa, at condimentum augue cursus quis. Nullam sed ante sit amet libero hendrerit posuere eget a dolor. 

\chapter{Titolo del capitolo 2}
\input{chapters/cap2}

\chapter{Conclusioni}
\input{chapters/conclusioni}

% QUI È POSSIBILE INSERIRE LA BIBLIOGRAFIA

\begin{thebibliography}{}
\bibitem{google} 
www.google.com
\bibitem{facebook} 
www.facebook.com

\end{thebibliography}













\end{document}